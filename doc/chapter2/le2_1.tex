\section{Le 2.1}

Consider the following DAE:
\begin{align*}
    E\,\dot x &= A\,x + B\,u, & y &= C\,x
\end{align*}
There exists invertible matrices $P$ and $T$, where $x = T\,w$ and multiplying the model equations with $P$ from the left and $T,$ $T^{-1}$ to the right of $E$ gives 
\begin{align*}
    P\,E\,T\,T^{-1}\dot x &= P\,A\,T\,T^{-1}\,x + P\,B\,T\,T^{-1}\,u. \\
    P\,E\,T\,\dot w &= P\,A\,T\,w + P\,B\,T\,T^{-1}\,u,
\end{align*}
where
\begin{align*}
    P\,E\,T = \begin{pmatrix}
        I & 0 \\ 0 & E_2 
    \end{pmatrix} \quad P\,A\,T = \begin{pmatrix}
        A_1 & 0 \\ 0 & I
    \end{pmatrix}, \quad \text{where $E_2$ is nilpotent.}
\end{align*}
% Note that $P\,E\,T \equiv E$, and $P\,A\,T \equiv A$.

Assuming that the ODE part, $\dot w_1 = A_1\,w_1 + B_1\,u$ is observable, i.e., 
\begin{align*}
    \begin{pmatrix}
        \lambda\,I - A_1 \\ C
    \end{pmatrix} \quad \text{has full column rank } \forall \lambda, \text{ i.e., } \begin{pmatrix}
        \lambda\,I - A_1 \\ 0 \\ C
    \end{pmatrix} \text{ also has full column rank } \forall \lambda.
\end{align*}

Considering the following:
\begin{align*}
    \begin{pmatrix}
        \lambda\,E - A \\ C
    \end{pmatrix} &\equiv \begin{pmatrix}
        \lambda\,P\,E\,T - P\,A\,T \\ C
    \end{pmatrix} = \begin{pmatrix}
        \lambda\,\begin{pmatrix}
            I & 0 \\ 0 & E_2 
        \end{pmatrix} - \begin{pmatrix}
            A_1 & 0 \\ 0 & E_2 
        \end{pmatrix} \\ \begin{pmatrix}
            C_1 & C_2
        \end{pmatrix}
    \end{pmatrix}
    = \begin{pmatrix}
        \lambda\,I - A_1 & 0 \\ 0 & \lambda\,E_2 - I \\ C_1 & C_2
    \end{pmatrix}
\end{align*}
Any square matrix has a Jordan normal form if the field of coefficients is extended to one containing all the eigenvalues of the matrix [\emph{from Wikipedia}].
The Jordan form is given as 
\begin{align*}
    \begin{pmatrix}
        \lambda_1 & 1 \\ & \lambda_1 & 1 \\ & & \lambda_1 \\ & & & \lambda_2 & 1 \\ & & & & \lambda_2 \\ & & & & & \lambda_3 \\ & & & & & & \ddots \\ & & & & & & & \lambda_n 
    \end{pmatrix}, \text{for a nilpotent matrix}, \lambda_1 = \lambda_2 = \cdots = \lambda_n = 0
\end{align*}
Thus, $\lambda\,E_2 - I$ has always full-column rank. 

\fbox{
    \parbox{0.95\textwidth}{
        Therefore, if $\begin{pmatrix}
            \lambda\,I - A_1 & C
        \end{pmatrix}^T$ has full-column rank then $\begin{pmatrix}
            \lambda\,E - A & C
        \end{pmatrix}^T$ has full-column rank $\forall\ \lambda$. \\
        
        If $\begin{pmatrix}
            \lambda\,E - A & C
        \end{pmatrix}^T$ has full-column rank then its minor $\begin{pmatrix}
            \lambda\,I - A_1 & C
        \end{pmatrix}^T$ also has full-column rank $\forall\ \lambda$.
    }
}


% The DAE can be partitioned as
% \begin{align*}
%     E_1\,\dot x_1 &= A_1\,x_1 + B_1\,u, & E_2\,\dot x_2 &= A_2\,x_2 + B_2\,u, & y &= \begin{pmatrix}
%         C_1 & C_2
%     \end{pmatrix}\,x
% \end{align*}
% with $E_2\,\dot x_2 = A_2\,x_2 + B_2\,u,$ beging the ODE, i.e., usually $E_2 = I$.
% \begin{align*}
%     \begin{pmatrix}
%         E_1 & 0 \\ 0 & E_2
%     \end{pmatrix}\,\dot x &= \begin{pmatrix}
%         A_1 & 0 \\ 0 & A_2
%     \end{pmatrix}\,x + \begin{pmatrix}
%         B_1 \\ B_2
%     \end{pmatrix}\,u & y &= \begin{pmatrix}
%         C_1 & C_2
%     \end{pmatrix}\,x
% \end{align*}

% Assume that the ODE part of the system is observable, i.e.,
% \begin{align*}
%     \text{rank}\begin{pmatrix}
%         \lambda E_1 - A_1 \\ C_1
%     \end{pmatrix} = n
% \end{align*}

% The observability matrix for the entire system can be written as
% \begin{align*}
%     \mathcal{O}x &= \begin{pmatrix}
%         \lambda \begin{pmatrix}
%             E_1 & 0 \\ 0 & E_2
%         \end{pmatrix} - \begin{pmatrix}
%             A_1 & 0 \\ 0 & A_2
%         \end{pmatrix} \\ \begin{pmatrix}
%             C_1 & C_2
%         \end{pmatrix}
%     \end{pmatrix}
%     = \begin{pmatrix}
%         \lambda E_1 - A_1 & 0 \\ 0 & \lambda E_2 - A_2 \\ C_1 & C_2
%     \end{pmatrix}
% \end{align*}
% \begin{theorem}
%     If $A$ is an $n \times m$ matrix, then the determinant of a $p \times p$ submatrix obtained from $A$ is called a minor of order $p$.
% \end{theorem}

% The submatrix is $\begin{pmatrix}
%     \lambda E_1 - A_1
% \end{pmatrix}$ and has the order $p$, with $p$ number of 'states'. Since this is an ODE, this matrix is non-zero

% The submatrix is $\begin{pmatrix}
%     \lambda E_2 - A_2
% \end{pmatrix}$ and has the order $q$, with $q$ number of 'states'. Since this is part of a DAE this has the set of algebraic expressions which is also not a non-zero matrix.

% For a SISO system, $C_1$ and $C_2$ is not a square matrix.

% % The largest minor is $\begin{pmatrix}
% %     \lambda E_2 - A_2 \\ C_2
% % \end{pmatrix}$.

% % $\det\begin{pmatrix}
% %     \lambda E_2 - A_2
% % \end{pmatrix}$ is non-zero since the ODE is observable.

% \begin{theorem}
%     The rank of an $n \times m$ matrix is equal to the order of its largest non-zero minor.
% \end{theorem}

% % The rank of $\det\begin{pmatrix}
% %     \lambda E_2 - A_2 \\ C_2
% % \end{pmatrix}$ is the equal to the order of $\det\begin{pmatrix}
% %     \lambda E_2 - A_2
% % \end{pmatrix}$, which is equal to the number of states $n$.

% \begin{theorem}
%     A tall $n \times m$ matrix is full column is \underline{if and only if} there exists a non-zero minor of order $p$.
% \end{theorem}

% % Therefore,...
% 1