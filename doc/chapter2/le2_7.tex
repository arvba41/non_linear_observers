\section{Le 2.7}
Consider the system,
\begin{align*}
    \dot x &= f(x) + g(x)\,u & y &= h(x)
\end{align*}
differentiating $y(t)$ one time 
\begin{align*}
    \dot y(t) &= \frac{\partial h(x)}{\partial x}\,\dot x = \frac{\partial h(x)}{\partial x}\,\left(f(x) + g(x)\,u\right) = L_f\,h(x) + L_g\,h(x)\,u.
\end{align*}
Similarly, 
\begin{align*}
    L_f^2\,h(x) &= \frac{\partial L_f\,h(x)}{\partial x}\,f(x), & L_g\,L_f\,h(x) &= \frac{\partial L_f\,h(x)}{\partial x}\,g(x)
\end{align*}
The system is of relative degree equal to the number of states $n$, then 
\begin{align*}
    L_g\,L_f^k\,h(x) &= 0, \forall\ x\ \in\ \mathcal{N}(x^0)\ \forall\ k < n - 1 & L_g\,L_f^{n-1}\,h(x) &\neq 0.
\end{align*}
Considering the observability criteria, 
\begin{align*}
    \mathcal{O} &= \frac{\partial}{\partial x}\begin{pmatrix} 
        h(x) \\ L_f\,h(x) \\ \vdots \\ L_f^{n-1}h(x) 
    \end{pmatrix} = \begin{pmatrix}
        L_f\,h(x) \\ L_f^2\,h(x) \\ \vdots \\ L_f^{n}h(x) 
    \end{pmatrix}.
\end{align*}
The rank of $\mathcal{O}$ is equal to $n$, the number of states if the relative degree equal to the number of states.
